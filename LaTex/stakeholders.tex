\documentclass[AdvProjMgmt_Sebastien_Deriaz]{subfiles}


\begin{document}
\section{Analyse des parties prenantes}
Les parties prenantes sont les suivantes :
\begin{enumerate}
\item Professeur encadrant le projet
\item Entreprise partenaire
\item Étudiant / développeur
\item Utilisateur du projet
\item Contributeur
\end{enumerate}
\paragraph{Professeur} : L'objectif du professeur est de guider l'étudiant pour l'amener à faire les bons choix et s'assurer que le projet soit sur la bonne voie
\paragraph{Entreprise partenaire} : L'entreprise cherche à obtenir un produit fonctionnel qui rempli le cahier des charges. L'orientation open-source du projet ne profite pas à l'entreprise mais plutôt aux utilisateurs privés et aux écoles.
\paragraph{Étudiant / développeur} : L'étudiant doit mener à bien le projet et garantir qu'il rempli le cahier des charges. En parallèle il doit s'assurer que les objectifs de l'entreprise, même si ils ne sont pas explicitement décris dans le cahier des charges, soient rempli dans le cadre du projet (par exemple modifier le système pour prendre en compte un point évident qui n'aurait pas été mentionné dans le cahier des charges)
\paragraph{Utilisateur du projet} : L'utilisateur a un rôle critique et va fournir un feedback utile pour déceler les défauts de conception du système autant sur le plan ergonomie que fiabilité.
\paragraph{Contributeur} : Le/Les contributeur(s) agissent dans un second temps, dans le cas où le système est fonctionnel. Ils améliorent le système par le biais de l'open-source en fournissant des morceaux de codes permettant d'étendre les capacités du système.\\
\subsection{Organisation}
La gestion de projet sera, dans un premier temps, consacrée uniquement aux 4 premières parties prenantes du projet, afin de mener à bien la création d'un prototype.



\end{document}