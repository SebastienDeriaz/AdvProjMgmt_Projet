\documentclass[AdvProjMgmt_Sebastien_Deriaz]{subfiles}


\begin{document}
\section{Workpackages}
Les workpackages pour le projet sont les suivants
\begin{enumerate}[leftmargin=2cm,label={WP\arabic*  :}]
\item Gestion de projet
\item Analyse des protocoles / Analyse des appareils
\item Conception du prototype : Implémentation et réalisation
\item Tests (par le développeur et par un agent externe)
\item Gestion des contributeurs (open-source)
\end{enumerate}
\paragraph{Gestion de projet} : Affectation du travail, création du planning et coordination des parties prenantes
\paragraph{Analyse des protocoles et appareils} : Recherche de l'état de l'art en terme de communication et de contrôle dans un premier temps. Dans un deuxième temps il s'agit de déterminer si et comment les protocoles de communication peuvent être implémentés dans le programme
\paragraph{Conception de prototype} : Il s'agit de développer la base du système sur laquelle s'ajoute la couche de gestion de la communication et des protocoles, développée dans la phase d'analyse.
\paragraph{Tests} : Les tests sont séparés en deux parties : tests par le développeur pour vérifier que l'implémentation des protocoles est fonctionnelle et tests par l'utilisateur qui permettront de valider l'ergonomie et la fiabilité du système.
\paragraph{Gestion des contributeurs} : Cette partie se déroule après que le projet ait été livré. Il s'agit de gérer le travail effectué bénévolement par chaque contributeur, notamment quelles parties sont publiques et modifiables et quelles parties du programme sont fixes. Il convient également de s'assurer que le travail effectué par chacun respecte une logique de développement harmonieuse








\end{document}